\section{Versuchsaufbau/-durchführung}
Ziel des verwendeten Aufbaus ist das Photolumineszenzlicht verschiedener Nanokristalle in Abhängigkeit
von der Anregungswellenlänge, der Intensität und der Polarisation spektral zu untersuchen.
Hierzu wird eine Anordnung nach Abbildung~\ref{fig: aufbau} verwendet. Die einzelnen Komponenten
im Strahlengang sollen nun näher beschrieben werden.

\begin{figure}
  \centering
  \includegraphics[width = 0.4\textwidth]{pics/setup_pl.png}
  \caption{Bild des verwendeten Aufbaus zur Aufnahme von Photolumineszenzspektren kolloidaler
  Nanostrukturen \cite{anleitung_pl}.}
  \label{fig: aufbau}
\end{figure}

Zunächst wird als Anregungslichtquelle eine Laserdiode mit $\lambda = \SI{405}{\nano\meter}$ verwendet.
Das ausgesandte Licht passiert eine optische Diode, die Rückkopplungseffekte minimieren soll.
Anschließend folgt ein einstellbarer Neutraldichtefilter, mit dem die auf einen Maximalwert eingestellte Laserleistung
verringert werden kann. Die Leistung lässt sich mit einem optional einsetzbaren Leistungsmessgerät bestimmen.
Anschließend gelangt das Laserlicht auf die Probe. Hierbei handelt es sich um Nanokristalle in Pulverform, die sich in
einem durchsichtigen Kunststoffgefäß befinden. Unter einem Winkel von $\SI{20}{\degree}$ zur optischen Achse wird der
Detektionspfad installiert. Das gestreute Licht wird zunächst mit einer Sammelinse eingefagen. Zur Überprüfung auf eine
mögliche Polarisation des Photoluminszenzlichtes wird eine Kombination aus $\lambda / 2$-Plättchen und
Glan-Taylor-Prisma verwendet. 
Abschließend wird das Licht mit einer Linse auf den Eingang eines Faser-gekoppelten
Spektrometers fokussiert. Mit einer Computer-Steuerung können Spektren aufgenommen werden.

Es werden Photolumineszenzspektren von drei verschiedenen Proben aufgenommen. Anschließend wird eine Abhängigkeit von der
Leistung und von der Polarisation überprüft. Abschließend werden andere Anregungswellenlängen verwendet. Hierzu wird
eine Weißlicht-Quelle eingesetzt, aus der mit optischen Filtern die Wellenlängen
$\SI{532}{\nano\meter}$ und $\SI{633}{\nano\meter}$ extrahiert werden.
