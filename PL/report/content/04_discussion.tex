\section{Diskussion}
Die Abschätzung der Größe der Potentialtöpfe in Abschnitt \ref{sec:schatz} ist ohne Probleme
für die Proben mit den Emissionswellenlängen $\lambda_{\text{pl}} = \SI{645}{\nano\meter}$
und $\lambda_{\text{pl}} = \SI{580}{\nano\meter}$ möglich. Bei der $\lambda_{\text{pl}} = \SI{520}{\nano\meter}$
Probe muss darauf geachtet werden die Leistung deutlich niedriger zu wählen, da die Photolumineszenz der Probe
so intensiv ist, dass die Diode in Sättigung geht. Daraus kann auf eine stark erhöhte Photolumineszenz für
Anregungen nahe der Resonanz geschlossen werden. Die Abweichung der gemessenen Photolumineszenzwellenlänge für die $\lambda_{\text{pl}} = \SI{580}{\nano\meter}$ und die $\lambda_{\text{pl}} = \SI{520}{\nano\meter}$ Proben könnte darauf hinweisen, dass die Herstellerangaben nicht korrekt sind.\\
Die Sättigung der Diode tritt ebenfalls für die in \ref{sec:leistung}
gemachte Messung der Photolumineszenz der $\lambda_{\text{pl}} = \SI{520}{\nano\meter}$ Probe ab
einer Leistung von $\SI{4}{\milli\watt}$ auf. Dieser Messpunkt wird daher in der folgenden Diskussion nicht berücksichtigt.\\
\\
Anhand der Messungen zur Leistungsabhängigkeit in Abschnitt \ref{sec:leistung} lässt sich ein
durch die Probe auftretender Sättigungseffekt erkennen. Bei genügend hoher Eingangsleistung
konvergiert die Photolumineszenz gegen einen Maximalwert, da alle Ladungsträger im Kristall
angeregt sind. Auch hier ist zu erkennen, dass für Anregungen nahe der Resonanz bereits
deutlich kleinere Leistungen ausreichen.\\
Die Schwankungen der Position der maximalen Emissionsenergie in Abbildung \ref{fig:posem}
lassen sich möglicherweise durch im Material stattfindende, intensitätsabhängige, nichtlineare
Effekte erklären. Für die unterschiedliche Art der Abhängigkeit für die verschiedenen Proben konnte
kein spezifischer Grund festgestellt werden.\\
Das konvergierende Verhalten der Halbwertsbreiten in \ref{fig:FWHM} ist ebenfalls durch
die Sättigung der Probe zu erklären. Die grundsätzliche spektrale Verbreiterung der PL
gegenüber dem anregenden Licht ist auf leicht unterschiedliche Quantenpunktgrößen
innerhalb einer Probe zurückzuführen. Für die Variation von Intensität,
spektraler Breite und Position der Lumnineszenzpeaks sind außerdem die
bereits in Kapitel \ref{sec:theo} erläuterten Effekte verantwortlich. \\
\\
Aus den Messungen bei unterschiedlich polarisiertem anregendem Licht
(Abschnitt \ref{sec:Polarisation}) ist keine Abhängigkeit der Photolumineszenz
von der Polarisation zu erkennen. Dies lässt sich durch die zufällige Orientierung
der Quantenpunkte im Probenbehälter erklären, wodurch eine im Mittel unpolarisierte Photolumineszenz entsteht.\\
\\
Die in Abschnitt \ref{sec:weisslicht} ausgewerteten Messungen bei anderen Anregungsenergien
zeigen ein stark unterschiedliches Verhalten der Photolumineszenz
mit variierendem Verhältnis von Anregungs- und Photolumineszenzwellenlänge.
Für die $\lambda_{\text{pl}} = \SI{645}{\nano\meter}$ Probe ist keine Photolumineszenz
vor dem Hintergrundrauschen auszumachen. \\
Insgesamt weisen die mithilfe der Weisslichtquelle erstellten Spektren deutlich breitere
Photolumineszenzpeaks auf als die auf Laseranregung basierenden. Dies kann durch die
höhere spektrale Breite des aus der Weisslichtquelle gefilterten, anregenden Lichtes erklärt werden.\\

Zusammenfassend lässt sich sagen, dass der Aufbau gut geeignet ist um die Größen der
Quantenpunkte mithilfe von Photolumineszenzmessungen zu bestimmen, sowie um unterschiedliche
Abhängigkeiten der Photolumineszenz von Quantenpunkten vom anregenden Licht qualitativ festzustellen.
%
% Sättigung der diode bei anregung nahe der resonanz(520 probe 405 lambda)
% Sättigung bei 520 520 weisslicht
% bi und triexcitonen verhalten der Emissionsmaximapositionen
% sättigung der diode
% sättigung der Probe
% reflexion
%
% verschiebung der wellnlänge nonlinear
% breite und intensität der emmissionspeaks je nach Anregungswellenlänge
% kolloidal => keine polarisationsabhängigkeit (statistik)
% spektrale breiten, wegen doch leicht unterschiedlicher Strukturgrößen und einer mittleren PL
