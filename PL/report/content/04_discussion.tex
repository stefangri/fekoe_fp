\newpage
\section{Diskussion}
Die Abschätzung der Größe der Nanostrukturen in Abschnitt \ref{subsec:31} ist ohne Probleme
für die Proben mit den Emissionswellenlängen $\lambda_h = \SI{645}{\nano\meter}$
und $\lambda_h = \SI{580}{\nano\meter}$ möglich. Bei der $\lambda_h = \SI{520}{\nano\meter}$
Probe muss darauf geachtet werden, die Leistung deutlich niedriger zu wählen, da die Photolumineszenz der Probe
so intensiv ist, dass die Diode in Sättigung geht. Daraus kann auf eine stark erhöhte Photolumineszenz für
Anregungen nahe der Resonanz geschlossen werden. Die Abweichung der gemessenen Photolumineszenzwellenlänge für die $\lambda_h = \SI{520}{\nano\meter}$ Probe deuten auf eine falsche Herstellerangabe auf der Probe hin. Aus den Messungen in Abschnitt \ref{subsec:31} und \ref{subsec:34} ergibt sich eine ungefähre Emissionswellenlänge von ca. $559\,\si{\nano\meter}$.\\
\\
Anhand der Messungen zur Leistungsabhängigkeit in Abschnitt \ref{sec:leistung} lässt sich ein
durch die Probe auftretender Sättigungseffekt erkennen. Dies deutet daraufhin, dass die Generation von Exzitonen nicht mehr linear von der Intensität des eingestrahlten Lichts abhängt und für die Erzeugung eines Exzitons mehr als ein Photon absorbiert wird. Weiterhin ist es möglich, dass mehrere Exzitonen gleichzeitig in einem Quantenpunkt enstehen. Da diese miteinander Wechselwirken verlängert sich die Lebenszeit der angeregten Zustände wodurch weniger neue Anregungen möglich werden. Durch die Präsenz mehrerer Exzitonen in einem Quantenpunkt bilden sich außerdem Biexzitonen, die Grund für die Veränderungen der Emissionswellenlänge sein können.\\
Die Schwankungen der Position der maximalen Emissionsenergie in Abbildung \ref{fig:posem}
lassen sich möglicherweise durch im Material stattfindende, intensitätsabhängige, nichtlineare
Effekte erklären. Die Verschiebung hin zu höheren Wellenlängen bei der $645\si{\nano\meter}$ Messung entsteht möglicherweise durch eine stattfindende Aufheizung der Probe. Die Verschiebung der Emissionswellenlänge resultiert dabei aus einer Vergrößerung der Bandlücke mit zunehmender Temperatur. \cite{thermik}
Die Schulter in Abbildung \ref{fig:pl520} ist ein Anzeichen für die typische inhomogene Verbreiterung der Photolumineszenz bei Messungen an unterschiedlich großen Nanostrukturen.\\
Das konvergierende Verhalten der Halbwertsbreiten in Abbildung \ref{fig:FWHM} ist ebenfalls durch
die Sättigung der Probe zu erklären. Die grundsätzliche spektrale Verbreiterung der PL
gegenüber dem anregenden Licht ist auf leicht unterschiedliche Quantenpunktgrößen
innerhalb einer Probe zurückzuführen. Für die Variation von Intensität,
spektraler Breite und Position der Lumnineszenzpeaks sind außerdem die
bereits in Kapitel \ref{sec:theo} erläuterten Effekte verantwortlich. \\
\\
Aus den Messungen bei unterschiedlich polarisiertem anregendem Licht
(Abschnitt \ref{sec:Polarisation}) ist keine Abhängigkeit der Photolumineszenz
von der Polarisation zu erkennen. Dies lässt sich durch die zufällige Orientierung
der Quantenpunkte im Probenbehälter erklären, wodurch eine im Mittel unpolarisierte Photolumineszenz entsteht.\\
\\
Die in Abschnitt \ref{subsec:34} ausgewerteten Messungen bei anderen Anregungsenergien
zeigen ein stark unterschiedliches Verhalten der Photolumineszenz
mit variierendem Verhältnis von Anregungs- und Photolumineszenzwellenlänge.
Für die $\lambda_h = \SI{645}{\nano\meter}$ Probe ist keine Photolumineszenz
vor dem Hintergrundrauschen auszumachen. \\
Insgesamt weisen die mithilfe der Weißlichtquelle erstellten Spektren deutlich breitere
Photolumineszenzpeaks auf als die auf Laseranregung basierenden. Dies kann durch die
höhere spektrale Breite des aus der Weißlichtquelle gefilterten, anregenden Lichtes erklärt werden. Das Verhältnis der spektralen Breiten von Laser und gefilterter Weißlichtanregung liegt bei ungefähr $1:8$. \\

Zusammenfassend lässt sich sagen, dass der Aufbau gut geeignet ist um die Größen der
Quantenpunkte mithilfe von Photolumineszenzmessungen zu bestimmen, sowie um unterschiedliche
Abhängigkeiten der Photolumineszenz von kolloidalen Nanostrukturen bei Raumtemperatur vom anregenden Licht qualitativ festzustellen.
%
% Sättigung der diode bei anregung nahe der resonanz(520 probe 405 lambda)
% Sättigung bei 520 520 weisslicht
% bi und triexcitonen verhalten der Emissionsmaximapositionen
% sättigung der diode
% sättigung der Probe
% reflexion
%
% verschiebung der wellnlänge nonlinear
% breite und intensität der emmissionspeaks je nach Anregungswellenlänge
% kolloidal => keine polarisationsabhängigkeit (statistik)
% spektrale breiten, wegen doch leicht unterschiedlicher Strukturgrößen und einer mittleren PL
