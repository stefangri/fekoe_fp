\section{Versuchsdurchführung}
Mit dem in \autoref{subsec:afmfunk} beschriebenen Aufbau werden im Folgenden Scanlinien verschiedener Proben aufgenommen. Dafür wird die z-Piezo Nachregelung des Rasterkraftmikroskops aktiviert. Zuerst wird eine Silliziumprobe vermessen, auf der kreis- und linienförmige, sowie quadratische Strukturen aufgebracht sind. Die Scans werden bei einer Auflösung von $250 \times 250$ Pixeln auf einer Fläche von $20\times\SI{20}{\micro\meter}$ aufgenommen. Als Scangeschwindigkeit wurden 100 pps gewählt.
Anschließend werden die Speicherpits einer CD und einer DVD vermessen. Bei der CD wird eine Fläche von $10\times\SI{10}{\micro\meter}$ und bei der DVD von $5\times\SI{5}{\micro\meter}$ vermessen. Scangeschwindigkeit und Auflösung bleiben gleich. Als Höhenkalibrierung wird für alle Scans ein Faktor von $\SI{155}{\nano\meter\per\volt}$ eingestellt. Für jede Messung wird ein Vorwärts- und Rückwärtsscan durchgeführt.\\
\\
Abschließend werden für die Materialien Teflon, Edelstahl und Kohlenstoff in Diamantstruktur (DLC) Kraft Abstandskurven aufgenommen. Dafür wird die z-Piezo Nachregelung abgeschaltet. Die Starthöhe wird dabei so gewählt, dass eine vollständige Kraft-Abstands-Kurve aufgenommen werden kann. 
