\newpage
\section{Diskussion}
Die Messungen an den Mikrostrukturen zeigen, dass diese hervorragend geeignet sind, die Apparatur auf
mögliche Fehlerquellen zu untersuchen. Durch Profilmessungen entlang der beiden lateralen Achsen,
können Asymmetrien festgestellt werden. Insbesondere die Betrachtung der Kreisstruktur eignet sich
gut, um System-bedingte Verzerrungen zu veranschaulichen. Bereits bei der Messung der Mikrostrukturen,
die wesentlich größer sind, als die später vermessenen Speicher-Disks, fällt auf, dass die
Daten insbesondere in der $y$-Richtung stark verrauscht sind. Dies deutet auf eine deformierte
Cantilever-Spitze hin. Es ist empfehlenswert anhand der Messungen an der Mikrostruktur zu entscheiden,
ob die Spitze vor den weiteren Messungen gewechselt werden sollte. Zur genaueren Analyse der vorliegenden
Spitzendeformation könnten noch feinere Strukturen verwendet werden.

Trotz der verminderten Auflösung konnten die relevanten Abmessungen einer CD gemessen werden und realistische
Ergebnisse für die Speicherkapazität gewonnen werden. Zur ausreichend guten Auflösung einer
DVD und Blu-Ray reichte die Qualtät des Aufbaus nicht aus.

Auch zur Vermessung von elastischen Eigenschaften eignet sich der Aufbau, wie aus der Analyse der
Kraft-Abstands-Kurven hervorgeht. Auch wenn z.B. die beiden bestimmten Werte des Elastizitätsmoduls
deutlich voneinander abweichen, konnte zumindest die richtige Größenordnung bestimmt werden. Gleiches gilt
für die Messung der auftretenden Adhäsionskräfte.

Zusammenfassend lässt sich sagen, dass der Aufbau durch seine leichte Handhabung sehr gut zur schnellen
Untersuchung von Mikrostrukturen geeignet ist. In Kombination
mit einer guten Cantilever-Spitze könnten vermutlich noch kleinere Strukturen ($\sim \si{\nano\meter}$) vermessen werden.
