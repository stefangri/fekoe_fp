\section{Auswertung}
Im Folgenden werden die Messungen präsentiert und nötige Rechnungen durchgeführt. Zunächst
werden die Kalibrierungen der lateralen Position vorgestellt, mit denen aus einem Spannungssignal der
Viersegment Diode eine Position auf der Probe angegeben werden kann. Dies wird genutzt um anschließend
aus den Messungen an den Quarzkugeln und Polysterenkugeln die Boltzmann-Konstante, sowie die leistungsabhängige
Fallensteifigkeit zu bestimmen. Abschließend werden die Messungen und Beobachtungen an den Zwiebelzellen
gezeigt.

\subsection{Kalibrierung der lateralen Position}


\subsection{Quarzkugeln}
Die Messungen an den Quarzkugeln wurden für die Pumpstromstärken $\SI{70}{\milli\ampere}$, $\SI{170}{\milli\ampere}$,
$\SI{270}{\milli\ampere}$, $\SI{370}{\milli\ampere}$ und $\SI{470}{\milli\ampere}$ durchgeführt. Da das Vorgehen der Auswertung
jeweils völlig analog funktioniert, werden hier nur im Detail die Messungen für $I = \SI{70}{\milli\ampere}$ gezeigt.

Abbildung~\ref{}~(a) und (c) zeigen die Kugelpositionen in $x$- und $y$-Richtung unter Einwirkung
einer periodischen Kraft () durch die Piezosteuerung des Probentisches.
Aus diesen Daten wird als Schätzer für die spektrale Leistungsdichte
ein Periodogramm berechnet. Dies ist jeweils in den Abbildungen~\ref{}~(b) und (d) einzusehen. Die Kurven werden
an eine Funktion der Gestalt
\begin{equation}
  PSD(f) = \frac{A}{f^2 + f_0^2}
\end{equation}
angepasst. Hierin ist $f_0$ die \emph{roll-off}-Frequenz. Es ergibt sich
\begin{equation}
  f_{0, \SI{70}{\milli\ampere}} \approx
\end{equation}
Aus dieser Frequenz kann gemäß
\begin{equation}
  k =
\end{equation}
die Fallensteifigkeit $k$ ermittelt werden. Abbildung~\ref{} zeigt die Abhängigkeit der Fallensteifigkeit von dem
Pumpstrom $I$ und damit von der Laser-Intensität. Es ist ein linearer Trend zu erkennen, wobei der Datenpunkt $k_x$
für $\SI{370}{\milli\ampere}$ deutlich ausreißt.

In den Abbildungen~\ref{}~(a) und~(c) ist die Kugelposition als Funktion der Zeit ohne Krafteinwirkung dargestellt.
Es ist sowohl in $x$- als auch in $y$-Richtung eine Asymmetrie in der Verteilung der Positionen mit der Zeit zu erkennen.
Dies ist deutlicher in einer histogrammierten Ansicht (Abbildung~\ref{}~(b) und (c)) zu beobachten.
Aus den Daten kann die Varianz $\langle x^2 \rangle \overset{\langle x \rangle = 0} = \langle x\rangle^2$ ermittelt werden
\begin{equation}
  \langle x^2 \rangle
\end{equation}
Und hieraus unter Verwendung des Äquipartionstheorems ein Wert für die Boltzmannkonstante
\begin{equation}
k_b
\end{equation}
Der seit diesem Jahr festgeegte Wert der Boltzmann-Konstante beträgt . Unabhängig von den Messungen unter Krafteinwirkung
kann mit gegebener Boltzmann-Konstante die Fallensteifigkeit berechnet werden. Für $I = \SI{70}{\milli\ampere}$ ergibt sich
exemplarisch
\begin{equation}

\end{equation}
Die Werte weichen deutlich von denen in~\eqref{} ab.

\subsection{Polysterenkugeln}
Vollkommen analog zu der Auswertung aus dem vorangegenagen Abschnitt wird für die Polystrenkugeln vorgegangen.
Abbildung~\ref{} zeigt die Leistungsabhängigkeit der Fallensteifigkeit. Im Mittel aaus allen
Berechnungen wird für die Boltzmannkonstante der Wert
\begin{equation}
  k_b
\end{equation}
gefunden.

\subsection{Zwiebelzellen}
