\section{Diskussion}
\label{sec: diskussion}

Mit dem durchgeführten Versuch konnte die Funktionsweise einer optischen Pinzette
hervorragend beobachtet werden. In dem Video 470mA\_quarz.avi im Quarz-Anhang ist beispielsweise
deutlich zu erkennen, wie eine einzelne Quarzkugel mit der Pinzette eingefangen wurde, während sich
drumherum andere Kugeln diffus bewegen. Auch die Anwendung der optischen Pinzette in der Zellbiologie
konnte nachvollzogen werden. Innerhalb einer Zelle ist es möglich die Bewegung einzelner Vesikel
zu kontrollieren. Weitere Messungen könnten hier auch quantitative Ergebnisse leifern, z.B. für die
notwendige Kraft um die Bewegung eines Vesikels zu stoppen.

Die Messungen unter äußerem Krafteinfluss haben keine Ergebnisse für die Fallensteifigkeit geliefert, was
an einer falschen Versuchsdurchführung liegt. Die Einstellung der Frequenz und Amplitude, und
damit der Geschwindigkeit $v$, der treibenden
Schwingung wurde für alle Stromstärken weitgehend konstant gehalten. Um vergleichbare Messwerte zu erhalten,
sollte mit steigender Fallensteifigkeit $k$, gemäß des Kräftegleichgewichtes $kx = \beta v$ auch die
Geschwindigkeit erhöht werden.

Ohne äußeren Krafteinfluss konnten realistische Ergebnisse für die Fallensteifigkeit bestimmt werden. Der
Verlauf der spektralen Leistungsdichten gemäß einer Lorentzfunktion bestätigt den Brownschen Charakter der
Teilchenbewegung. Insbesondere die Messung mit den Polysterenkugeln liefert einen Wert für die Boltzmann-Konstante, der
in der gleichen Größenordnung wie der tatsächliche Wert liegt. Der Vergleich der Abbildungen~\ref{fig: quarz_k_power_series}
und~\ref{fig: poly_k_power} zeigt deutlich auf, dass für eine zuverlässige Messung lediglich ein Teilchen in der Nähe der
Falle sein darf um Störungen der gemessenen Diodensignale zu minimieren.
In beiden Messungen ist eine Asymmetrie zwischen $x$- und $y$-Richtung der Fallensteifigkeit gemessen worden. Dies könnte
an einer tatsächlichen Asymmetrie oder an einem systematischen Fehler der Piezosteuerung liegen.
