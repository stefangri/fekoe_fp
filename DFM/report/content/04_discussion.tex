\section{Diskussion}
Die Ergebnisse des voran gegangenen Abschnitts haben gezeigt, dass der vorhanden Aufbau hervorragend geeignet ist
um mikroskopische Dunkelfeldaufnahmen von plasmonischen Nanostrukturen mit der CMOS Kamera zu erstellen. Das Prinzip
der Dunkelfeldmikroskopie, mit dem sich Strukturen unterhalb des Beugungslimits auflösen lassen, konnte daher
demonstriert werden.

Zur Untersuchung der spektralen Eigenschaften des gestreuten Lichtes, ist der Aufbau bisher nur bedingt geeignet.
Wie die Spektren aus Abbildung~\ref{fig: au_röhren_fits} zeigen, ist der Einfluss des weißen Lichtes zu groß. Der Einsatz einer
größeren Blende, würde den Aufbau erheblich verbessern und eventuell sogar eine gänzliche Filterung des Streulichtes
ermöglichen.

Dennoch konnte in den Spektren der Nanoröhren (Abbildung~\ref{fig: au_röhren_fits}) eine eindeutige
polarisationsabhängigkeit beobachtet werden (vgl. etwa maximale Intensität in \ref{fig: au_röhren_fits} c) und d)).
Das Modell~\eqref{eq: fit_q_sc}, das die Vorhersage der Mie-Theorie enthält, ist kompatibel mit den
gemessenen Daten. Insbesondere konnte mit dem Fit eine Polarisationsabhängigkeit der Halbwertsbreite und der Position der
Amplitude der Streupeaks gefunden werden. Um die Ergebnisse zu verifizieren, müsste
jedoch der Einfluss der Weißlichtlampe unterdrückt werden.
