\section{Auswertung}

\subsection{\ce{Au}-Nanoröhren}
\label{sec: auröhren}
In den Abbildungen~\ref{fig: au_röhren_bilder} a)-d) sind vier beispielhafte Aufnahmen der
\ce{Au}-Nanoröhren mit der CMOS-Kamera einzusehen.
Zudem sind in den Abbildungen~\ref{fig: au_röhren_fits} a) und b) zwei weitere Aufnahmen gezeigt, für die
ebenfalls ein Spektrum aufgenommen wurde. Hierbei wurde die selbe Stelle der
Probe einmal mit einer Einstellung des Polarisationsfilter von $\SI{104}{\degree}$
und einmal von $\SI{14}{\degree}$ (um $\SI{90}{\degree}$ gedreht) aufgenommen.
Die Spektren sind in Abbildung~\ref{fig: au_röhren_fits} c) und d) gezeigt, wobei
hier bereits das Spektrum einer Dunkelaufnahme abgezogen wurde. Es ist deutlich zu erkennen, dass
die maximal Intensität bei der Einstellung von $\SI{14}{\degree}$ höher ist. Die Spektren sollen
nun quantitativ analysiert werden.
\begin{figure}
  \centering
  \includegraphics[scale = 1]{../analysis/results/pics_röhren.png}
  \caption{Dunkelfeld Aufnahmen der \ce{Au}-Nanoröhren, aufgenommen mit der CMOS Kamera.}
  \label{fig: au_röhren_bilder}
\end{figure}

Zunächst soll der Einfluss von ungestreutem Licht, das in das Spektrometer gelangt ist, untersucht werden.
Hierzu wird das normierte Spektrum $S(\lambda)$ der Weißlichtquelle (Daten aus~\cite{thorlabs} entnommen)
mittels des Modells
\begin{equation}
  D(\lambda) = A \cdot S(\lambda), \quad A \in R^{+}
  \label{eq: fit_light}
\end{equation}
an die Daten $D(\lambda)$ angepasst. Das Ergebnis für die beiden Spektren ist ebenfalls in den Abbildungen~\ref{fig: au_röhren_fits} c) und d)
eingezeichnet (grüne Linien). Der Fit macht deutlich, dass die Gestalt der aufgenommenen Spektren
durch das Licht der Lichtquelle dominiert wird.

Dennoch soll die Kompatibilität mit dem Mie-Modell überprüft werden. Hierzu wird als alternatives Modell zur Beschreibung der
Daten eine Linearkombination aus dem Spektrum der Lampe und dem Streuquerschnitt $Q_{sc}$ angesetzt
\begin{equation}
  D(\lambda) = A \cdot S(\lambda) + B \cdot Q_{sc}(\lambda, d).
  \label{eq: fit_q_sc}
\end{equation}
Der Durchmesser der Nanopartikel $d$ wird hierbei ebenfalls als freier Fit-Parameter zugelassen.
Für die Funktion ist ebenfalls der Wellenlängen-abhängige komplexe Brechungsindex $n + ik$ von Gold relevant. Hierzu wurden
Daten aus~\cite{ref_index_au} entnommen und stückweise, quadratisch interpoliert, sodass Werte für alle hier auftretenden Wellenlängen
vorhanden sind. Die Verläufe der Brechungsindix-Komponenten und die Interpolationen sind in Abbildung~\ref{fig: ref_index}
dargestellt. Die Fits an das Modell~\eqref{eq: fit_q_sc} sind in Abbildung~\ref{fig: au_röhren_fits} eingezeichnet.
Für die Durchmesser $d$ ergeben sich für die beiden Polarisationseinstellungen
\begin{equation}
  d_{104} \approx \SI{90}{\nano\meter}, \quad d_{14} \approx \SI{101}{\nano\meter}.
\end{equation}
Nach~\cite{anleitung} haben die Nanoröhren eine Ausdehnung von $\SI{}{\nano\meter}$.
In Abbildung~\ref{fig: compare_q_sc} sind die Komponenten~$Q_{sc}(\lambda, d)$ der Fits zum Vergleich erneut aufgetragen. Hier sind ebenfalls
die Halbwertbreiten der Peaks eingezeichnet. Für die Position $\lambda_{max}$ der maximalen Intensitäten gilt
\begin{equation}
  \lambda_{max, 14} \approx \SI{598}{\nano\meter}, \quad \lambda_{max, 104} \approx \SI{582}{\nano\meter}
\end{equation}
und für die Halbwertsbreiten $\Delta \lambda$
\begin{equation}
  \Delta \lambda_{14} \approx \SI{139}{\nano\meter}, \quad \Delta \lambda_{104} \approx \SI{108}{\nano\meter}.
\end{equation}
%{'max_I': 19.533990965687238, 'max_wl': 598.1981981981983, 'lam+': 679.0790790790791, 'lam-': 539.7397397397398, 'fwhm': 139.3393393393393}
%{'max_I': 17.688797357477288, 'max_wl': 582.982982982983, 'lam+': 645.0450450450451, 'lam-': 537.3373373373373, 'fwhm': 107.70770770770775}
Hieraus kann eine Abschätzung für die Lebenszeit $\tau$ der Plasmonen vorgenommen werden
\begin{equation}
  \tau_{14} \approx \SI{1}{\second}, \quad   \tau_{104} \approx \SI{1}{\second}.
\end{equation}


\begin{figure}
  \centering
  \includegraphics[scale = 1]{../analysis/results/refrective_index.png}
  \caption{Verlauf der Komponenten $n$ und $k$ des komplexen Brechungsindices $n + i k$
  von Gold mit den Daten nach \cite{ref_index_au} und einer quadratischen Interpolation.}.
  \label{fig: ref_index}
\end{figure}


\begin{figure}
  \centering
  \includegraphics[scale = 1]{../analysis/results/fit_14_104.png}
  \caption{a) und b) CMOS Aufnahmen der Stellen auf der Probe, für die die Spektren in c) und d)
  gemessen wurden. Neben den Messdaten sind in c) und d) die Ergebnisse der Fits an die Modelle~\eqref{eq: fit_q_sc}
  und~\eqref{eq: fit_light} eingezeichnet.}
  \label{fig: au_röhren_fits}
\end{figure}

\begin{figure}
  \centering
  \includegraphics[scale = 1]{../analysis/results/compare_14_104.png}
  \caption{Vergleich der gefitteten Streuquerschnitte $Q_{sc}$ für die beiden Polarisationen $14$ und $\SI{104}{\degree}$.}
  \label{fig: compare_q_sc}
\end{figure}
\subsection{\ce{Au}-Nanosphären}
Abbildung~\ref{fig: au_sphären_bilder} zeigt zwei der aufgenommenen CMOS Bilder an der Probe mit den \ce{Au}-Nanosphären.
Es wurde für eine feste Einstellung der Polarisation ein Spektrum des Streulichtes, sowie ein
Dunkelspektrum gemessen. Die Daten sind in Abbildung~\ref{fig: au_sphären_data} aufgetragen. In Teil c) ist die Differenz
der Spektren dargestellt, die dem reinen Einfluss der Nanosphären entsprechen sollte.
Hier zeigt sich, dass die Spektren in etwa gleich verlaufen.
Dies spricht dafür, dass in dem Bereich der Probe, für den das Dunkelspektrum aufgenommen wurde, Nanospähren
vorhanden waren.
Auf eine quantitative Analyse der Spektren wie in Abschnitt~\ref{sec: auröhren} wird daher verzichtet.

\begin{figure}
  \centering
  \includegraphics[scale = 1]{../analysis/results/pics_spheres.png}
  \caption{Dunkelfeld Aufnahmen der \ce{Au}-Nanosphären, aufgenommen mit der CMOS Kamera.}
  \label{fig: au_sphären_bilder}
\end{figure}

\begin{figure}
  \centering
  \includegraphics[scale = 1]{../analysis/results/fit_spheres.png}
  \caption{Aufgenommene Spektren der \ce{Au}-Nanosphären a) und b), sowie die Differenz der Kurven c).}
  \label{fig: au_sphären_data}
\end{figure}
